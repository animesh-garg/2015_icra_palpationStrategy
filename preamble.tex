\IEEEoverridecommandlockouts
\overrideIEEEmargins

%floats and figures
\usepackage{graphics}
\usepackage[pdftex]{graphicx}
\usepackage[font={small}]{caption}
\usepackage{subcaption}
%\usepackage[center]{subfigure} %DONT USE BOTH SUBCAPTION AND SUBFIGURE
%\DeclareGraphicsExtensions{.pdf,.png,.jpg}
%\usepackage{overpic}
%\usepackage[rightcaption]{sidecap}
%\usepackage{pbox}

%Math Stuff
\usepackage{mathtools}
\usepackage{amsmath, amssymb, amscd}
%\usepackage{ wasysym } %special symbols
\usepackage{amsfonts}
\usepackage{mathptmx}       % selects Times Roman as basic font
\DeclareMathAlphabet{\mathcal}{OMS}{lmsy}{m}{n}
\DeclareSymbolFont{largesymbols}{OMX}{cmex}{m}{n}
\usepackage{algorithm}
\usepackage{algorithmicx}
%\usepackage{algorithm}
%\usepackage{algpseudocode}
% \usepackage[ruled,vlined,linesnumbered]{algorithm2e}
\usepackage{ textcomp } %for getting text tilde

%Table Stuff
\usepackage{array} %for table entries to be in center of cell
\usepackage{tabularx}
\usepackage{multicol}
\usepackage{multirow}

%DOCUMENT WIDE
\usepackage{times} % assumes new font selection scheme installed
\usepackage{xspace}
\usepackage[english]{babel} %for hyphenation rules
\usepackage{flushend}%balance columns on last page
\usepackage{fixltx2e} %fix latex issue across versions
\usepackage{bm}
\usepackage{units}

\usepackage{makeidx}
\usepackage{enumitem}
\usepackage[yyyymmdd,hhmmss]{datetime}
\usepackage[english]{babel}

%Bibliography and cross-ref
\makeatletter
\let\NAT@parse\undefined
\makeatother
\usepackage[numbers]{natbib}
\renewcommand{\bibfont}{\footnotesize}
% \usepackage{cite} %DONT USE NATBIB AND CITE TOGETHER

%hyperlinking
\usepackage{url}
\makeatletter
\g@addto@macro{\UrlBreaks}{\UrlOrds}
\makeatother
\usepackage{color}
\usepackage[usenames,dvipsnames, table]{xcolor}
\usepackage[pdfborder={0 0 0.5}]{hyperref}
\hypersetup{
    colorlinks=true,
    linkcolor=blue,
    citecolor=black,
    filecolor=cyan,
    urlcolor=blue
}


%=======U S E R  D E F I N E D  M A C R O S=======
% \newcommand{\bibhref}[2]{#2}
\newcommand{\todo}[1]{\textcolor{red}{[#1]}}
\newcommand{\tocite}[1]{\textcolor{red}{[cite]}}
\newcommand{\ignore}[1]{}

% Usage:
% \figlabel{myfigure} creates \label{fig:myfigure}
% \figref{myfigure} references it
\newcommand{\figlabel}[1]{\label{fig:#1}}
\newcommand{\figref}[1]{Figure~\ref{fig:#1}}

% Usage:
% \seclabel{mysection} creates \label{sec:mysection}
% \secref{mysection} references it
\newcommand{\seclabel}[1]{\label{sec:#1}}
\newcommand{\secref}[1]{Section~\ref{sec:#1}}

% Usage:
% \tablabel{mytable} creates \label{tab:mytable}
% \tabref{mytable} references it
\newcommand{\tablabel}[1]{\label{tab:#1}}
\newcommand{\tabref}[1]{Table~\ref{tab:#1}}

% use this command instead of writing "da Vinci" so it's never split 
\newcommand{\davinci}{da~Vinci\xspace}
